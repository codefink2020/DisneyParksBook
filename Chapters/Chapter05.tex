% Chapter X

\chapter{TimeLine of disneyLamnd} % Chapter title

\label{ch:tickets} % For referencing the chapter elsewhere, use \autoref{ch:name}

%----------------------------------------------------------------------------------------

\section{1955}
\subsection{}






Disneyland’s Main Street USA 1955 Almost Ready!

1954-2005
\section{1954}
July 16
More than 180 acres of South Anaheim Orange Groves begin their transformation into the “Happiest Place on Earth.”  To finance Disneyland, Walt Disney sells a vacation home and borrows against his life insurance.

August 13:
Excavation of the Disneyland site begins.

October 27:
“The Disneyland Story” airs (first episode of Disneyland television show).
\section{1955}
January 15 1955
Despite plans to open the park without it, Walt decides to go ahead and build Tomorrowland, saying that the park can only open as a complete experience.  Admiral Fowler advised that the outlook for Tomorrowland was very grave, and despite the warnings, Tomorrowland was completed in six months and opened with the Autopia, Space Station X-1, and the Monsanto Hall of Chemistry.

July 17 1955
Walt and Lillian’s 30th Anniversary celebrated at Disneyland.

Mark Twain makes its first circle of the Rivers of America.

The two Disneyland steam trains (C.K. Holliday and E.P. Ripley) make their first run around the Park.

Disneyland opens with 18 attractions, including the Jungle Cruise, Tomorrowland, Autopia, Mr. Toad’s Wild Ride, and the Mark Twain.

The televised opening is hosted by Ronald Reagan, Art Linkletter, and Bob Cummings.

The 11,000 invitation only tickets to opening day are so easily duplicated that first day attendance shoots to 28,154, with an ABC television audience of 90 million.

One entrepreneur folds a ladder over the park’s backside fence and lets people in for $5.

Official general admission cost $1. Cost of attractions ranged from 10c to 35c

September 1955
One Millionth Guest Admitted (Elsa Marquez)

October 1955
Disneyland Hotel opens on 60 acre site next to the Park.
Ticket books available for the first time, containing A, B, and C Tickets.

November 1955
On Thanksgiving, the Mickey Mouse Club Circus debuts with a parade down Main Street USA but is poorly received and discontinued in early 1956.

December 1955
Walt Disney leads first Christmas Parade

\section{1966}
January 1956
Thirteen attractions open, including Tom Sawyer Island and the Skyway.  Traditional “Fantasy in the Sky” fireworks display debuts.  D Tickets added to Ticket Books.  5 Millionth Guest Admitted January 1957

Sleeping Beauty Castle’s interior walkways, Frontierland Shooting Gallery and Fantasyland Autopia, where riders race at a top speed of 11 mph, are among eight new attractions.  Fantasy in the Sky Fireworks debut, 25 pack mules bought for \$ 50 a piece.
\section{1957}
December 1957
10 Millionth Guest Admitted (Leigh Woolfenden)
\section{1958}
January 1958
Alice in Wonderland, Grand Canyon Diorama, Main Street Fire trucks, and the three masted Sailing Ship Columbia debut.

January 18: The Disneyland Kennel opens.

\section{1959}
January 1959
15 Millionth Guest Admitted.

June 1959
E Tickets added to Ticket Books.

January 1960
America the Beautiful, Nature’s Wonderland (based on Disney’s True-Life Adventure movies), Art of Animation debut.
20 Millionth Guest Admitted

January 21: The Disneyland special “A Gala Day at Disneyland” was released. It showcased the opening of the Submarines, Monorail, and Matterhorn.

April 1961
25 Millionth Guest Admitted (Dr. Glenn C. Franklin) June, Tinker Bells First Flight.

December 1961
Former President Eisenhower and his wife visit Disneyland.

January 1962
Six tons of steel and 110 cubic yards of concrete become the Swiss Family Treehouse. Also opening are the Safari Game Shoot, the Pavilion Restaurant, and the Tahitian Terrace. “Audio animatronic” elephants arrive in the Jungle Cruise.

June 1963
Tiki Room Opens

January 1964
Fantasyland Theatre Opens

January 1965
Great Moments with Mr. Lincoln comes to Disneyland. The audio-animatronic Lincoln production highlights the park’s 10th anniversary.

July 1966
New Orleans Square opens, the park’s first new “land.”

December 1966
At 9:35 AM December 15, Walt Disney, 65, dies of cancer.  Disney had been recovering from surgery a month earlier to remove one of his lungs.  His brother, Roy, takes charge of the Disney entertainment empire.

January 1967
Pirates of the Caribbean premieres.  Tomorrowland opens six new attractions, including Carousel of Progress, Flight to the Moon, and a redesigned Rocket Jets.  First daily operating PeopleMover system in the United States begins operation.

January 1968
Mark III monorails debuted. January 21: “Pirates of the Caribbean to the World of Tomorrow” aired.  This special took viewers on their first ride through Pirates of the Caribbean and gave a tour of the newly reopened Tomorrowland.  It featured lots of behind the scenes footage of the construction of both.

August 1969
Haunted Mansion Opens

July 1970
Disneyland’s 15th Anniversary Celebration.

August 1970
Hippies protesting the Vietnam War invade the park August 6. Some parade down Main Street USA and speak of “liberating” Minnie Mouse.  Others take over Fort Wilderness on Tom Sawyer’s Island, yank down the 15 star American flag and raise a Viet Cong flag in its place.  The protest escalates when the park is ordered closed at 7 PM and 30,000 guests are escorted out.  Trash cans are set on fire and rocks and debris are thrown at police blocking the main gate.  Police make 23 arrests.

December 1970
At 11:45 PM December 20, Roy Disney, 78, dies from a cerebral hemorrhage.

June 1971
100 Millionth Guest Admitted. (Valerie Suldo)

January 1972
Main Street Electrical Parade, which uses 500,000 twinkling lights, winds its way through the park for the first time.  Bear Country becomes park’s seventh theme “land.”  Attractions include Davy Crockett’s Explorer Canoes, Indian Trading Post, and Country Bear Jamboree.

January 1973
Walt Disney Story opens on Main Street USA.  The exhibit includes memorabilia and pictures from the family’s archives and a short film on the park founder’s life.

January 1974
America Sings, a review of the nation’s musical history, premieres in Tomorrowland’s rotating Carousel Theater.

January 1975
Mission to Mars, an updated version of Flight to the Moon, opens. America on Parade, honoring the nation’s bicentennial, premieres.

December 1976
Gorrilla’s in Camp added to the Jungle Cruise Attraction.

January 1977
Space Mountain opens after two years of construction. The $20 million attraction costs more than Disney spent to build the entire park in 1955.

January 1978
Abominable Snowman added to the Matterhorn, November. Mickey Mouse turns 50 with a November celebration attended by 91,762 guests.

January 1979
The two acre Big Thunder Mountain Railroad opens in Frontierland.

January 1982
The E ticket remains in the language but disappears from Disneyland.  The passport, good for admission and unlimited use of park’s attractions, is introduced at a cost of $12 for adults.  July-Disneyland Band performs for the 50,000th time.

January 1983
Revamped Fantasyland opens, adding Pinocchio’s Daring Journey.

August 1985
250 Millionth Guest Admitted. (Brooks Charles Arthur Burr)

January 1986
The 17-minute 3D film “Captain EO,” featuring Michael Jackson, premieres in Tomorrowland.  Big Thunder Ranch, a recreation of an 1880s horse ranch, opens in Frontierland.

January 1987
Star Tours is launched.  George Lucas combines his “Star Wars” characters and space flight simulator technology to bring the attraction to life.  Herbert Ryman brought out of retirement to draw conceptual sketches and themes for Indana Jones Adventure.  Premier of Disney Dollars, First Disneyland State Fair.

January 1989
The 87-foot-tall Splash Mountain opens.  The ride features a five-story drop down a simulated waterfall.  September- 300 Millionth Guest Admitted. (Claudine Masson)

January 1992
Fantasmic, Disney’s $30 million fire, water and laser show, debuts in the Rivers of America.

January 1993
Disneyland’s $100 million Toontown opens, the first new land since Bear Country in 1972.  The cartoon village features Roger Rabbit’s Toontown Spin, the Jolly Trolly, Donald Duck’s boat and Chip ‘n’ Dale’s treehouse.

Work begins in the parking lot of Disneyland.  The monorail is rerouted about 40 feet closer to the center of the parking lot and some 10 foot deep holes are dug in the middle of a new fenced off area. A new ride based on the Indiana Jones character is being built.

The Jungle Cruise is shut down for refurbishment.  Sign announcing Indiana Jones Adventure is put on guard wall.  Work begins on Indiana Jones Adventure Queue area.

When Monorail is up running again, pilots ad lib spiel about latest greatest Adventureland attraction opening spring 1995.

January 1994
Disneyland records its 350 Millionth Guest.

January 1995
The $100 million Indiana Jones Adventure debuts.  Guests board World War II transports for a ride past bubbling lava pits, mummies, snakes and a giant rolling boulder.

July 1997
400 Millionth Guest Admitted. (Minnie Pepito)

May 1998
The new Tomorrowland opens.

August 1999
Disneyland introduces Fast Pass, a way to bypass long lines by getting a pass with a set return time.

January 2000
One Day Adult Ticket Prices increase from $39 to $41. Animator Marc Davis with Disney since 1935 dies on 1/12.  Marc played an active roll in the planning of Disneyland.  He developed story and character concepts for many Disneyland attractions, including Pirates of the Caribbean, the Haunted Mansion, and It’s A Small World. The Disneyland Kennel closes.

March 2001
March 16 – 450 Millionth Guest Admitted. (Mark Ramirez of Lytle, Texas). Mark Ramirez, a civilian production management specialist for the U.S. Air Force, was declared the 450 millionth visitor by Mickey Mouse just as he passed through the turnstile.  Ramirez was honored during a ceremony where he was presented with a lifetime pass to the Disneyland Resort (Disneyland and Disney’s California Adventure), a stay at the new Disney’s Grand Californian Hotel, and a gift pack.  He was also given the honor of changing the official attendance sign at the Disneyland park Train Station from 400 million to 450 million. “I’ll be able to point to the sign and say I made that change.  It’s great being a part of Disney history,” he said.  The family later rode in a cavalcade with Mickey Mouse down Main Street, U.S.A., led by the Disneyland Band.  In addition, the Ramirez family was treated to VIP tours of both Disneyland and the newest park Disney’s California Adventure.

April 2001
Disneyland announced that full-time city paramedics were being stationed at its Anaheim parks in case of emergencies.  Disney officials said they would beef up emergency services by stationing city paramedics full time at the park, a plan that will cost them $1.4 million annually and speed response time.

July 2001
Disneyland turns 46

December 2001
Walt Disney turns 100

July 2005 Disneyland turns 50!

%------------------------------------------------

\subsection{Subsection Title}

Content

%------------------------------------------------

\subsection{Subsection Title}

Content

%----------------------------------------------------------------------------------------

\section{Section Title}

Content